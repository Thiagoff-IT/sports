\documentclass[12pt, a4paper]{article}

% Pacotes fundamentais
\usepackage[utf8]{inputenc}
\usepackage[T1]{fontenc}
\usepackage[brazil]{babel}
\usepackage{graphicx}
\usepackage{geometry}
\usepackage{titlesec}
\usepackage{hyperref}
\usepackage{listings}
\usepackage{xcolor}
\usepackage{indentfirst}
\usepackage{float}

% Configuração das margens
\geometry{
    top=3cm,
    bottom=2cm,
    left=3cm,
    right=2cm
}

% Configuração de links
\hypersetup{
    colorlinks=true,
    linkcolor=black,
    filecolor=magenta,      
    urlcolor=blue,
    citecolor=black,
}

% Configuração para códigos C++
\definecolor{codegreen}{rgb}{0,0.6,0}
\definecolor{codegray}{rgb}{0.5,0.5,0.5}
\definecolor{codepurple}{rgb}{0.58,0,0.82}
\definecolor{backcolour}{rgb}{0.95,0.95,0.92}

\lstdefinestyle{mystyle}{
    backgroundcolor=\color{backcolour},   
    commentstyle=\color{codegreen},
    keywordstyle=\color{magenta},
    numberstyle=\tiny\color{codegray},
    stringstyle=\color{codepurple},
    basicstyle=\ttfamily\footnotesize,
    breakatwhitespace=false,         
    breaklines=true,                 
    captionpos=b,                    
    keepspaces=true,                 
    numbers=left,                    
    numbersep=5pt,                  
    showspaces=false,                
    showstringspaces=false,
    showtabs=false,                  
    tabsize=2,
    language=C++
}

\lstset{style=mystyle}

% Informações do documento
\title{Relatório de Projeto: Sistema de Gestão Esportiva}
\author{Nome do Aluno}
\date{\today}

\begin{document}

% Capa
\begin{titlepage}
    \begin{center}
        \vspace*{1cm}
        
        \Large
        \textbf{NOME DA INSTITUIÇÃO}
        
        \vspace{1.5cm}
        
        \large
        Nome do Curso \\
        Disciplina: Programação Avançada
        
        \vspace{3.5cm}
        
        \textbf{\LARGE Sistema de Gestão Esportiva}
        
        \vspace{3.5cm}
        
        \large
        \textbf{Nome do Aluno}
        
        \vfill
        
        \large
        Cidade - Estado \\
        2025
        
    \end{center}
\end{titlepage}

% Sumário
\tableofcontents
\newpage

% Conteúdo
\section{Introdução}
Este relatório documenta o desenvolvimento de um sistema de gestão esportiva em C++. O objetivo principal do software é facilitar a administração de torneios, permitindo o cadastro e gerenciamento de equipes, jogadores e partidas, além da geração automática de rankings.

O sistema foi projetado com foco na modularidade, utilizando Orientação a Objetos para garantir a extensibilidade e manutenção do código.

\section{Modelagem do Sistema}
Esta seção apresenta os diagramas que ilustram a estrutura e o comportamento do sistema.

\subsection{Diagrama de Classes}
O Diagrama de Classes abaixo representa a estrutura estática do sistema, mostrando as classes, seus atributos, métodos e relacionamentos.

\begin{figure}[H]
    \centering
    % \includegraphics[width=0.8\textwidth]{diagrama_classes.png}
    \fbox{\begin{minipage}{10cm}
        \centering
        \vspace{2cm}
        [Inserir Imagem do Diagrama de Classes Aqui]
        \vspace{2cm}
    \end{minipage}}
    \caption{Diagrama de Classes do Sistema}
    \label{fig:diagrama_classes}
\end{figure}

As principais relações identificadas são:
\begin{itemize}
    \item Um \textbf{Torneio} é composto por várias \textbf{Equipes}.
    \item Uma \textbf{Equipe} possui múltiplos \textbf{Jogadores}.
    \item Uma \textbf{Partida} associa duas \textbf{Equipes}.
\end{itemize}

\subsection{Diagrama de Casos de Uso}
O Diagrama de Casos de Uso ilustra as interações dos usuários com o sistema.

\begin{figure}[H]
    \centering
    % \includegraphics[width=0.8\textwidth]{diagrama_casos_uso.png}
    \fbox{\begin{minipage}{10cm}
        \centering
        \vspace{2cm}
        [Inserir Imagem do Diagrama de Casos de Uso Aqui]
        \vspace{2cm}
    \end{minipage}}
    \caption{Diagrama de Casos de Uso}
    \label{fig:casos_uso}
\end{figure}

\section{Funcionamento do Sistema}
O sistema opera através de um fluxo contínuo de interação com o usuário e processamento de dados.

\subsection{Fluxo de Execução}
O ciclo de vida da aplicação segue as seguintes etapas:

\begin{enumerate}
    \item \textbf{Inicialização}: O sistema carrega os dados previamente salvos (jogadores, equipes, rankings) através da classe \texttt{Persistencia}.
    \item \textbf{Menu Principal}: É apresentado ao usuário um menu de opções para navegação (ex: Cadastrar Equipe, Iniciar Torneio, Ver Ranking).
    \item \textbf{Processamento}: Conforme a escolha do usuário, os objetos são instanciados e manipulados em memória. Por exemplo, ao realizar uma partida, o sistema atualiza as estatísticas das equipes envolvidas.
    \item \textbf{Encerramento}: Ao sair, o sistema invoca os métodos de salvamento para garantir que as alterações sejam persistidas em arquivos (CSV ou binários).
\end{enumerate}

\subsection{Gerenciamento de Dados}
A persistência é realizada através de arquivos locais. O sistema lê arquivos CSV (como \texttt{jogadores.csv} e \texttt{ranking.csv}) na inicialização para popular as estruturas de dados em memória.

\section{Detalhamento das Classes}
Abaixo são descritas as responsabilidades das principais classes implementadas.

\subsection{Classe Jogador}
Representa a entidade atômica do sistema.
\begin{itemize}
    \item \textbf{Responsabilidade}: Manter dados como nome, idade e estatísticas individuais.
    \item \textbf{Destaque}: Implementa métodos para atualização de performance.
\end{itemize}

\subsection{Classe Equipe}
Agregadora de jogadores.
\begin{itemize}
    \item \textbf{Responsabilidade}: Gerenciar o elenco e calcular a força total do time.
    \item \textbf{Relação}: Possui um vetor de objetos \texttt{Jogador}.
\end{itemize}

\subsection{Classe Partida}
Representa o confronto.
\begin{itemize}
    \item \textbf{Responsabilidade}: Registrar o resultado, data e validar se as equipes estão aptas para o jogo.
\end{itemize}

\subsection{Classe Torneio}
O controlador principal da lógica de negócio.
\begin{itemize}
    \item \textbf{Responsabilidade}: Criar o chaveamento, agendar partidas e declarar o vencedor.
\end{itemize}

\subsection{Classe Ranking}
Responsável pela classificação.
\begin{itemize}
    \item \textbf{Responsabilidade}: Ordenar as equipes com base em critérios como pontos, vitórias e saldo de gols.
\end{itemize}

\subsection{Classe Persistencia}
Camada de acesso a dados.
\begin{itemize}
    \item \textbf{Responsabilidade}: Abstrair a leitura e escrita de arquivos, convertendo texto em objetos e vice-versa.
\end{itemize}

\section{Implementação}
Trechos de código relevantes que demonstram a lógica utilizada.

\begin{lstlisting}[caption={Exemplo de instanciação de Torneio}]
// Exemplo de uso no main.cpp
#include "src/Torneio.h"

int main() {
    Torneio torneio("Copa 2025");
    torneio.carregarDados();
    torneio.iniciarMenu();
    return 0;
}
\end{lstlisting}

\section{Conclusão}
O sistema desenvolvido atende aos requisitos de gestão esportiva, oferecendo uma solução robusta para o controle de torneios. A utilização de arquivos para persistência garante que os dados não sejam perdidos entre as execuções, e a estrutura orientada a objetos facilita futuras expansões, como a adição de novas modalidades esportivas.

\end{document}
